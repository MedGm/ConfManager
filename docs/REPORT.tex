\documentclass[12pt, a4paper]{report}
\usepackage[utf8]{inputenc}
\usepackage[T1]{fontenc}
\usepackage[french]{babel}
\usepackage{graphicx}
\usepackage{geometry}
\usepackage{hyperref}
\usepackage{booktabs}
\usepackage{float}
\usepackage{titlesec}
\usepackage{parskip}
\usepackage{xcolor}
\usepackage{listings}
\usepackage{fancyhdr}
\usepackage{caption}
\usepackage{subcaption}
\usepackage{longtable}
\usepackage{amsmath}

% Configuration de la mise en page
\geometry{hmargin=2.5cm,vmargin=2.5cm}

% Couleurs
\definecolor{primary}{RGB}{79, 70, 229} % Indigo
\definecolor{secondary}{RGB}{15, 23, 42} % Slate 900

% Configuration des liens
\hypersetup{
    colorlinks=true,
    linkcolor=primary,
    filecolor=magenta,      
    urlcolor=cyan,
    pdftitle={Rapport Final - Projet Gestion Conférence},
}

% Configuration des En-têtes et Pieds de page
\pagestyle{fancy}
\fancyhf{}
\fancyhead[L]{\leftmark}
\fancyhead[R]{\textbf{ConfManager}}
\fancyfoot[C]{\thepage}
\renewcommand{\headrulewidth}{0.4pt}
\renewcommand{\footrulewidth}{0.4pt}

% Style des chapitres
\titleformat{\chapter}[display]
  {\normalfont\bfseries\color{secondary}}{\chaptertitlename\ \thechapter}{20pt}{\Huge}

\begin{document}

%---------------------------------
% Page de garde
%---------------------------------
\begin{titlepage}
    \centering

    % --- Logos ---
    \makebox[\textwidth][s]{%
        \includegraphics[height=2.5cm, keepaspectratio]{images/logo_right.png}\hfill
        \includegraphics[height=2.6cm, keepaspectratio]{images/logo_left.png}
    }

    \vspace{2cm}

    {\large \textbf{Université Abdelmalek Essaâdi}}\\
    {\large \textbf{Faculté des Sciences et Techniques de Tanger}}\\
    {\large \textbf{Département Génie Informatique}}\\

    \vspace{2cm}

    \rule{\linewidth}{0.6mm}\\[0.3cm]
    {\huge\textbf{RAPPORT DE PROJET}}\\[0.2cm]
    {\Large\textbf{Plateforme de Gestion d'Événements (ConfManager)}}\\[0.3cm]
    \rule{\linewidth}{0.5mm}

    \vspace{0.8cm}

    \begin{figure}[H]
        \centering
        \includegraphics[width=0.4\linewidth, keepaspectratio]{images/logo.png}
    \end{figure}

    \vspace{2cm}
    
    \noindent
    \begin{minipage}[t]{0.48\textwidth}
        \raggedright
        {\large \textbf{Réalisé par (Équipe 7) :}}\\[0.3cm]
        \normalsize
        \textbf{Uthman Junaid} (Chef de Projet)\\
        Ahmane Yahya (Analyste)\\
        Essalhi Salma (Analyste)\\
        Kamouss Yassine (Dev Back)\\
        El Gorrim Mohamed (Dev Back)\\
        Salhi Mohamed (Dev Front)\\
        Kchibal Ismail (Testeur)\\
        Mohand Omar Moussa (Qualité)
    \end{minipage}
    \hfill
    \begin{minipage}[t]{0.48\textwidth}
        \raggedleft
        {\large \textbf{Encadré par :}}\\[0.3cm]
        \normalsize
        M. Boudhir Anouar Abdelhakim
    \end{minipage}

    \vspace{1cm}

    {\large Cycle Ingénieur – Logiciels et Systèmes Intelligents}\\[0.2cm]
    {\large Année Universitaire 2025–2026}\\

\end{titlepage}

\tableofcontents
\listoffigures
\newpage

% CHAPITRE 1 : INTRODUCTION
\chapter{Introduction Générale}

\section{Contexte du Projet}
Dans le milieu académique et professionnel, l'organisation de conférences nécessite une gestion rigoureuse des plannings, des inscriptions et de la communication. Notre projet \textbf{ConfManager} vise à développer une solution web centralisée, inspirée de plateformes comme EasyChair, pour simplifier ce processus.

\section{Objectifs}
L'objectif est de fournir une application permettant :
\begin{itemize}
    \item La création de comptes (Organisateurs / Participants).
    \item La gestion complète des événements (CRUD).
    \item Un système d'inscription fluide.
    \item La génération de plannings de conférences.
\end{itemize}

\newpage

% CHAPITRE 2 : GESTION DE PROJET
\chapter{Pilotage et Gestion de Projet}

Ce chapitre détaille l'organisation, la planification financière et temporelle, ainsi que l'analyse des risques du projet.

\section{Organisation de l'Équipe (OBS)}

\begin{table}[h]
\centering
\renewcommand{\arraystretch}{1.25}
\begin{tabular}{|p{3.5cm}|p{6cm}|p{5cm}|}
\hline
\textbf{Rôle} & \textbf{Membres} & \textbf{Responsabilités clés} \\ \hline
Chef de Projet &
UTHMAN JUNAID &
Pilotage du projet, planification, gestion des risques \\ \hline

Analystes &
AHMANE YAHYA \newline ESSALHI SALMA &
Analyse des besoins, WBS, cas d’utilisation \\ \hline

Développeurs Back-End &
KAMOUSS YASSINE \newline EL GORRIM MOHAMED &
API, base de données, logique métier \\ \hline

Développeur Front-End &
SALHI MOHAMED &
UI/UX, intégration front-end \\ \hline

Testeur &
KCHIBAL ISMAIL &
Tests unitaires et d’intégration \\ \hline

Qualité / Documentation &
MOHAND OMAR MOUSSA &
Revue de code, documentation \\ \hline
\end{tabular}
\caption{Organisation de l’équipe projet (OBS)}
\end{table}

\section{Structure de Découpage (WBS)}
Le projet a été découpé hiérarchiquement pour assurer une couverture complète des tâches.

\subsection*{1. Project Management}
\begin{itemize}
    \item \textbf{1.1 Initialization}: Scope, Team Setup, Tools Setup.
    \item \textbf{1.2 Planning}: WBS, Gantt, Budget, Risk Analysis.
    \item \textbf{1.3 Monitoring}: Weekly Reviews, Sprint Planning.
\end{itemize}

\subsection*{Outil de Gestion Agile (Interne)}
Conformément aux exigences, nous avons développé un module interne de gestion de projet (Agile Dashboard) directement intégré à la plateforme Back-Office. Cet outil permet de visualiser l'avancement des tâches par Sprint (Kanban View).

\begin{figure}[H]
    \centering
    \fbox{
        \includegraphics[width=\textwidth, keepaspectratio]{images/agile_board_screenshot.png}
    }
    \caption{Tableau de bord Agile intégré (Sprints \& Backlog)}
    \label{fig:agile_board}
\end{figure}


\subsection*{2. Analysis \& Design}
\begin{itemize}
    \item \textbf{2.1 Requirements}: User Stories (Functional/Non-Functional).
    \item \textbf{2.2 Modeling}: Use Case Diagrams, Class Diagrams.
    \item \textbf{2.3 UI/UX}: Wireframes (Figma), Prototype.
\end{itemize}

\subsection*{3. Development}
\begin{itemize}
    \item \textbf{3.1 Setup}: Next.js, Prisma.
    \item \textbf{3.2 Backend}: Auth (NextAuth), APIs (User, Event, Registration).
    \item \textbf{3.3 Frontend}: Landing Page, Dashboard, Event Details.
\end{itemize}

\subsection*{4. Testing \& Quality}
\begin{itemize}
    \item Unit Testing \& Integration Testing.
    \item User Acceptance Testing (UAT).
\end{itemize}

\subsection*{5. Closure}
\begin{itemize}
    \item Documentation (Dossier Technique).
    \item Présentation Finale.
\end{itemize}

\begin{figure}[H]
    \centering
    \includegraphics[width=0.8\textwidth, keepaspectratio]{images/wbs_diagram.png}
    \caption{Diagramme WBS}
\end{figure}

\section{Planification (Gantt)}
Le projet s'étale du 12 Décembre au 22 Janvier.

\begin{figure}[H]
    \centering
    \includegraphics[width=\textwidth, keepaspectratio]{images/gantt_chart.png}
    \caption{Planning Gantt du Projet}
\end{figure}

\section{Budget et Estimation des Charges}
Nous avons utilisé la méthode \textbf{COCOMO Simplifié}.

\subsection{Estimation}
\begin{itemize}
    \item \textbf{Taille estimée} : 2,000 lignes de code (2 KLOC).
    \item \textbf{Formule} : $Effort = 2.4 \times (KLOC)^{1.05} \approx 5$ mois-homme (Théorique).
    \item \textbf{Contexte Étudiant} : Ajusté à ~32 jours-homme effectifs (8 personnes x 4 semaines partiel).
\end{itemize}

\subsection{Budget Théorique}
% Basé sur un TJM fictif de 300€ :
$$ 32 \text{ jours} \times 300\text{€} = 9,600\text{€} $$

\section{Analyse des Risques}
Matrice Probabilité $\times$ Impact pour prioriser les risques.

\begin{longtable}{|c|p{5cm}|c|c|c|p{4cm}|}
\hline
\textbf{ID} & \textbf{Risque} & \textbf{Prob.} & \textbf{Imp.} & \textbf{PxI} & \textbf{Mitigation} \\ \hline
R1 & Retard dév (Tech stack) & 4 & 5 & \textbf{20} & Formation, Pair prog. \\ \hline
R2 & Indisponibilité membre & 3 & 4 & 12 & Binômage. \\ \hline
R3 & Mauvaise estimation & 4 & 3 & 12 & Agile, Revues hebdo. \\ \hline
R4 & Problèmes Git & 5 & 3 & 15 & Pull Requests strictes. \\ \hline
R5 & Effet Démo & 3 & 5 & 15 & Vidéo de secours. \\ \hline
\end{longtable}

\newpage

% CHAPITRE 3 : ANALYSE ET CONCEPTION
\chapter{Analyse et Conception}

\section{Analyse Fonctionnelle (Use Cases)}
Le système interagit avec 3 acteurs : Visiteur, Participant, Organisateur.

\begin{figure}[H]
    \centering
    \includegraphics[width=0.6\textwidth, keepaspectratio]{images/usecase_diagram.png}
    \caption{Diagramme des Cas d'Utilisation}
\end{figure}

\textbf{Scénarios Clés :}
\begin{itemize}
    \item \textbf{S'inscrire / Se connecter} : Pré-requis pour les rôles avancés.
    \item \textbf{Gérer événement} : Création, modification par l'organisateur.
    \item \textbf{S'inscrire à un événement} : Action principale du participant.
\end{itemize}

\section{Modélisation des Données (Classes)}
Le diagramme de classes reflète le schéma de base de données Prisma.

\begin{figure}[H]
    \centering
    \includegraphics[width=0.9\textwidth, keepaspectratio]{images/class_diagram.png}
    \caption{Diagramme de Classes}
\end{figure}

\textbf{Entités Principales :}
\begin{itemize}
    \item \texttt{User} : Stocke les infos d'authentification et le rôle.
    \item \texttt{Event} : L'entité centrale (dates, lieu, organisateur).
    \item \texttt{Session} : Sous-élément d'un événement (planning détaillé).
    \item \texttt{Registration} : Lien entre User et Event (Statut inscription).
\end{itemize}

\newpage

% CHAPITRE 4 : RÉALISATION
\chapter{Réalisation Technique}

\section{Architecture}
\begin{itemize}
    \item \textbf{Stack} : Next.js 14, TypeScript, Tailwind CSS, Prisma, SQLite (Dev).
    \item \textbf{Tests} : Jest \& React Testing Library (Unitaires \& Intégration).
\end{itemize}

\section{Interfaces Utilisateur}

\begin{figure}[H]
    \centering
    \begin{subfigure}{.49\textwidth}
      \centering
      \includegraphics[width=\linewidth]{images/screenshot_login.png}
      \caption{Page de Connexion}
    \end{subfigure}
    \begin{subfigure}{.49\textwidth}
      \centering
      \includegraphics[width=\linewidth]{images/screenshot_dashboard.png}
      \caption{Tableau de Bord}
    \end{subfigure}
    \caption{Aperçu de l'interface}
\end{figure}

\section{Tests et Validation}
Une suite de tests automatisés a été implémentée pour garantir la robustesse.
\begin{lstlisting}[language=bash, frame=single, caption={Exécution des tests avec Jest}]
PASS  src/components/events/RegisterButton.test.tsx
PASS  src/lib/utils.test.ts           
PASS  src/app/api/events/route.test.ts
Test Suites: 3 passed, 3 total
\end{lstlisting}

\section{Qualité de Code et Analyse Statique}
Pour assurer la maintenabilité et la sécurité du code, une intégration continue avec \textbf{SonarQube} a été mise en place. 

\subsection{Infrastructure}
L'environnement d'analyse repose sur une architecture conteneurisée via \textbf{Docker}, comprenant :
\begin{itemize}
    \item Un serveur SonarQube (v10.x).
    \item Une base de données PostgreSQL dédiée.
\end{itemize}

\subsection{Résultats de l'Analyse}
Le projet a été scanné à l'aide de \texttt{sonarqube-scanner}. L'analyse "Clean Code" a permis de détecter et corriger les "Code Smells" (variables inutilisées, complexité, duplication) et les vulnérabilités de sécurité.

\begin{figure}[H]
    \centering
    \includegraphics[width=\textwidth, keepaspectratio]{images/sonarqube_analysis.png}
    \caption{Rapport d'analyse SonarQube (Succès)}
    \label{fig:sonarqube}
\end{figure}

\newpage

% CONCLUSION
\chapter{Conclusion}
Le projet \textbf{ConfManager} a permis de mettre en œuvre une solution complète de gestion d'événements, respectant les standards du génie logiciel (WBS, UML, Tests). L'équipe a su surmonter les défis techniques (Next.js) et organisationnels pour livrer un produit fonctionnel dans les délais.

\end{document}
